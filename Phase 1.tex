\documentclass[12pt]{article}

\usepackage{standalone}
\usepackage{fullpage}

% rename 'part' to 'phase'
\renewcommand{\partname}{Phase}

\begin{document}
\part{Project Scoping}
\setcounter{section}{1}
\setcounter{subsection}{0}
\section*{Deliverable 1}
\subsection{GitHub - An Overview}
\textbf{GitHub} is a collaborative site for sharing and working on software projects based around the \textit{Git Version Control System}. It is beneficial when used individually or as a group as it allows for members to use a web-based graphical interface to interact with the command line tool git.\\ 

\noindent Github's advantages are its ability to provide easy interaction between members working on the same project, and its ease of use compared to working from the command line. Github allows for easy management of wikis, bug tracking and feature requests.\\

\noindent GitHub allows for the sharing of small code snippets (Gists) which act like normal repositories, so that they can be changed by the author (managed by Git version control on the website) and forked (making a separate copy of the project) by other users. Users are able to make changes to their projects locally and then push their changes to GitHub, but also can make changes directly on GitHub.

\vfill
\setcounter{section}{2}
\setcounter{subsection}{0}
\section*{Deliverable 2}
\subsection{Vision Statement}
To create a website which interfaces with GitHub's API that allows more advanced searching of repositories and improves a user's experience of viewing a project, both in an overview manner and in a more detailed view, linking relevant sections of code with documentation from the project's wiki
\subsection{General Goals}
??
\subsection{Group/Team Goals}
??
\subsection{Individual Team Member Goals}
\begin{enumerate}
\item Implementing methods for connecting functions from user code to the wiki. Also to oversee development of our project documentation using \LaTeX\ .
\item 
\item 
\item 
\item 
\end{enumerate}

\setcounter{section}{3}
\setcounter{subsection}{0}
\section*{Deliverable 3}
\subsection{Problem Statements}
\begin{enumerate}
\item GitHub doesn't have more advanced searching of repositories to find more readily contributable open source projects.
\item GitHub doesn't have a visual way to view a whole project (e.g. how all the classes/files interact, UML diagrams generated from code).
\item GitHub doesn't link documentation from a repository's wiki with the relevant functions/lines in the project's code.
\item GitHub doesn't allow functions/blocks of code to be collapsed for easy targeted viewing online.
\item GitHub doesn't show enough statistics (e.g. Average contributor input) to aid with choosing a new project.
\end{enumerate}
\subsection{Inference for Problem Statements}
??
\end{document}