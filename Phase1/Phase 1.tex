\documentclass[12pt]{article}

\usepackage{standalone}
\usepackage{fullpage}

% rename 'part' to 'phase'
\renewcommand{\partname}{Phase}

\begin{document}
\part{Project Scoping}
\setcounter{section}{1}
\setcounter{subsection}{0}
\section*{Deliverable 1}
\subsection{GitHub - An Overview}
\textbf{\textsf{GitHub}} is a collaborative site for sharing and working on software projects based around the \textit{Git Version Control System}. It is beneficial when used individually or as a group as it allows for members to use a web-based graphical interface to interact with the command line tool git.\\ 

\noindent \textsf{GitHub}'s advantages are its ability to provide easy interaction between members working on the same project, and its ease of use compared to working from the command line. It allows for easy management of wikis, bug tracking and feature requests.\\

\noindent The \textsf{GitHub} system also contains features for sharing small code snippets (Gists) which act like normal repositories, so that they can be changed by the author (managed by Git version control on the website) and forked (making a separate copy of the project). Contributors are able to make changes to their projects locally and then push their changes to \textsf{GitHub}, but also can make changes directly on the online \textsf{GitHub} service as well.


\setcounter{section}{2}
\setcounter{subsection}{0}
\section*{Deliverable 2}
\subsection{Vision Statement}
To create a website which interfaces with \textsf{GitHub}'s API allowing for a more advanced searching of repositories and improving a user's experience of viewing a project; both in an holistic perspective and a detailed view. We also aim to link relevant sections of code with documentation from the wiki and allow for an easier workflow for the user.
\subsection{General Goals}
\begin{itemize}
\item Expand the \emph{Explore GitHub} page and allow users to search for specific projects to increase contribution with a larger variety of open source projects.
\item Provide and display relevant information and statistics in an intuitive manner to aid users with choosing new open source projects.
\item Construct easy access between code and relevant documentation.
\item Create a simple and convenient way to navigate through documents, files and folders; allowing for users to view and contribute to specific sections easier.
\item Mediate a wider variety of resources (including statistics, code and projects) for utilisation.
\item Create a holistic view of projects to understand how systems interact with one another, allowing for deeper learning in the workings of current and new projects.
\end{itemize}
\subsection{Group/Team Goals}
\textbf{Statistics Team}: (Josh \& Daniel)
\begin{itemize}
\item Gather resources and statistics from current projects hosted on GitHub
\item Working with the \emph{Visual Team} to provide statistics which will shape the UI and workflow.
\end{itemize}
\textbf{Visual Team}: (Sanjay, Robert \& Paul)
\begin{itemize}
\item Develop a seamless User Interface (UI) that flows from the existing, to allow easy access to important data.
\item Creation of shortcuts to speed up common tasks (utilising data supplied from the \emph{Statistics Team}).
\end{itemize}
\subsection{Individual Team Member Goals}
\begin{enumerate}
\item Josh - Browse multiple repositories on \textsf{GitHub} in order to devise a set of statistics which consider a project to be be more readily contributable.
\item Daniel - Develop a system to aid users in selecting a project by displaying relevant statistics such as average input per contributor.  
\item Paul - Develop a visual based system to manage and see the interactions between project files and other projects hosted on \textsf{GitHub}.
\item Sanjay - Implementing methods for connecting functions from user code to the wiki. Also to oversee development of our project documentation using \LaTeX\ .
\item Robert - To provide a more convenient way of viewing code by implementing the collapsing of functions/code blocks.
\end{enumerate}

\pagebreak
\setcounter{section}{3}
\setcounter{subsection}{0}
\section*{Deliverable 3}
\subsection{Old Problem Statements}
At the current state, \textsf{GitHub}:
\begin{enumerate}
\item  Does not have more advanced searching of repositories to find more readily contributable open source projects.
\item Does not display enough statistics (e.g. Average input per contributor) to aid with choosing a new project.
\item \textbf{Does not  have a visual way to view a whole project (e.g. how all the classes/files interact, UML diagrams generated from code).}
\item Does not link documentation from a repository's wiki with the relevant functions/lines in the project's code.
\item \textbf{Does not allow functions/blocks of code to be collapsed for easy targeted viewing online.}
\end{enumerate}

\subsection{New Problem Statements}
At the current state, \textsf{GitHub}:
\begin{enumerate}
\item  Does not have more advanced searching of repositories to find more readily contributable open source projects.
\item Does not display enough statistics (e.g. Average input per contributor) to aid with choosing a new project.
\item Does not  have a visual way to view a whole project (e.g. how all the classes/files interact, collapsing of functions/blocks of code).
\item Does not link documentation from a repository's wiki with the relevant functions/lines in the project's code.
\item Does not have the ability to search for users (contributors) based on how active they are and what kind of projects they contribute to.
\end{enumerate}

\subsection{Explanation of Problem Statement Changes}
Problem statement 5 was found to be very specific and didn't allow enough room for creating features that could stem off of it. Also we thought of a new idea that GitHub did not have, the ability for authors' of repositories to search for other coders to contribute to their project. Subsequently, the old problem statement 5 was able to be easily merged with no. 3, giving it more breadth to allow for more features to be described.

\subsection{Inference for Problem Statements}
Inference for our problem statements stemmed from previous work and experience with \textsf{GitHub}. With a moderate level of knowledge of \textsf{GitHub}, we were able to identify a lack of features prevalent within usage of the system. Those of us who had lacked experience with usage of \textsf{GitHub} were quick to investigate and understand some important features to be missing.

Our first problem statement was brought up as a way to encourage younger or novice programmers to take part in contribution to open-source projects. Finding and aiding in projects that are large, cumbersome and complex are oft-not viable for programmers starting out in the field. Hence we proposed developing a more refined search feature that allows to find more readily contributable open source projects. Essentially, we would be enhancing resources available for the user.

The lack of a diverse set of statistics (such as average contribution per user) was a driving factor for our second problem statement, complementing our first problem statement. The availability of a greater set of statistical resources would be beneficial to aid individuals in finding suitable projects to work on. This was our inference for our second problem statement.

We also observed, explored and identified the absence of a visual aspect within the \textsf{GitHub} system. We realised the lack of a holistic view of code on \textsf{GitHub}. Previous work with software development prompted us to understand the need for a more visual aspect to viewing the relationships, connections and associations with other files in the repository.

The absence of a link between \textit{code}, (be it the functions within the code or only certain lines of code) and its respective \textit{documentation} was also another feature which we believe would he helpful. Prior work with coding and looking up relevant documentation for a function prompted us to list this fourth problem statement.

Our final statement was also another visual aspect. Once again, due to previous coding experience we realised the ability to collapse sections of code would provide a more pleasing, aesthetic look. As opposed to a large, clumsy, confusing code layout, we believe this feature would allow the end-user to focus their attention on a particular section.

All in all, we derived the essence of our problem statements based upon problems that we had previously encountered. We also formulated the problem statements for features and resources that we believed would enhance the user experience of the \textsf{GitHub} system.

\end{document}
