\documentclass[12pt]{article}

\usepackage{fullpage}
\usepackage{framed}
\usepackage{standalone}
\usepackage{amsmath}
\usepackage{tikz}

% rename 'part' to 'phase'
\renewcommand{\partname}{Phase}

% command to write requirement
\newcommand{\Requirement}[1] {
   \noindent \textbf{Requirement:} #1
}

% command to write feature
\newcommand{\Feature}[1]{ 
   \noindent \textbf{Feature} #1
}

% command to use template
\newcommand{\CFeature}[4]{
\noindent \textbf{Connextra Feature:}
	\begin{quote}
	\begin{tabular}{rl}
	\textbf{AS A} & #1\\
	\textbf{SO THAT \uppercase{#2}} & #3\\
	\textbf{\uppercase{#2} WANT TO} & #4  
	\end{tabular}
	\end{quote}
}

\newcommand{\GivenSc} {
	\noindent \textbf{GIVEN:}
	}
	
\newcommand{\WhenSc} {
	\noindent \textbf{WHEN:}
	}
	
\newcommand{\AndSc} {
	\noindent \textbf{AND:}
	}
	
\newcommand{\ThenSc} {
	\noindent \textbf{THEN:}
	}

\begin{document}

\documentclass[12pt]{article}

\usepackage{graphicx}
\usepackage{standalone}
\usepackage{fullpage}

% rename 'part' to 'phase'
\renewcommand{\partname}{Phase}

\begin{document}
\part{Project Scoping}
\setcounter{section}{1}
\setcounter{subsection}{0}
\section*{Deliverable 1}
\subsection{GitHub - An Overview}
\textbf{GitHub} is a collaborative site for sharing and working on software projects based around the \textit{Git Version Control System}. It is beneficial when used individually or as a group as it allows for members to use a web-based graphical interface to interact with the command line tool git.\\ 

\noindent Github's advantages are its ability to provide easy interaction between members working on the same project, and its ease of use compared to working from the command line. Github allows for easy management of wikis, bug tracking and feature requests.\\

\noindent GitHub allows for the sharing of small code snippets (Gists) which act like normal repositories, so that they can be changed by the author (managed by Git version control on the website) and forked (making a separate copy of the project) by other users. Users are able to make changes to their projects locally and then push their changes to GitHub, but also can make changes directly on GitHub.

\vfill
\setcounter{section}{2}
\setcounter{subsection}{0}
\section*{Deliverable 2}
\subsection{Vision Statements}
\subsection{General Goals}
\subsection{Group/Team Goals}
\subsection{Individual Team Member Goals}

\pagebreak
\setcounter{section}{3}
\setcounter{subsection}{0}
\section*{Deliverable 3}
\subsection{Problem Statements}
\begin{enumerate}
\item GitHub doesn't have more advanced searching of repositories to find more readily contributable open source projects.
\item GitHub doesn't have a visual way to view a whole project (e.g. how all the classes/files interact, UML diagrams generated from code).
\item GitHub doesn't link documentation from a repository's wiki with the relevant functions/lines in the project's code.
\item GitHub doesn't allow functions/blocks of code to be collapsed for easy targeted viewing online.
\end{enumerate}
\subsection{Inference for Problem Statements}
\end{document}

\pagebreak
\setcounter{part}{1}
\setcounter{section}{1}
\setcounter{subsection}{0}
\part{User Stories}
\section*{Deliverable 1}
\subsection{User Stories}


\begin{framed}
\subsubsection{Requirement: Advanced Search}
\noindent A more advanced searching of repositories to find more readily contributable open source projects.\\[0.2cm]

\hrule~\\

\noindent\Feature{\textbf{1:} An advanced search panel for finding repositories that I can commit to.}\\[0.2cm]


\CFeature{user (\textsf{GitHub} contributor)}{I}{can help other developers}{Find repositories that I can contribute to}

\noindent \textbf{Scenario}:
\begin{quote}
\begin{tabular}{rl}
\GivenSc & that I am on the main page\\
%\AndSc & lel\\
\WhenSc & I click on ``Find Repos"\\
\ThenSc & I can see a list of repositories with some basic stats next to each one
\end{tabular}
\end{quote}

\hrule~\\

\noindent\Feature{\textbf{2:} Filter repositories based on a number of parameters.}\\[0.2cm]

\CFeature{user (\textsf{GitHub} contributor)}{I}{can find more relevant repositories}{filter a list of repositories}

\noindent \textbf{Scenario}:
\begin{quote}
\begin{tabular}{rl}
\GivenSc & that I am on the main page\\
%\AndSc & lel\\
\WhenSc & I select some parameters and click on ``Find Repos''\\
\ThenSc & I can see a sorted list of repositories based on my parameters
\end{tabular}
\end{quote}
\end{framed}

\begin{framed}
\subsubsection{Requirement: Diverse set of Statistics}
Display enough statistics to aid with choosing a new project.\\[0.2cm]

\hrule~\\

\noindent\Feature{\textbf{3:} Repositories lists the average amount of commits per contributor}\\[0.2cm]

\CFeature{User (\textsf{GitHub} contributor)}{I}{Can view other contributors motivation}{Pick a repository that has high contribution rates}

\noindent \textbf{Scenario}:
\begin{quote}
\begin{tabular}{rl}
\GivenSc & that I am viewing a repository's statistics\\
%\AndSc & lel\\
\WhenSc & I click on ``Contributors"\\
\ThenSc & I can view the average commits per contributor
\end{tabular}
\end{quote}

\hrule~\\

\Feature{\textbf{4:} Repositories graphs average consecutive days for commits from contributors}\\[0.4cm]

\CFeature{User (\textsf{GitHub} contributor)}{I}{Know frequently I need to contribute}{View how frequently commits are made by contributors}

\noindent \textbf{Scenario}:
\begin{quote}
\begin{tabular}{rl}
\GivenSc & that I am viewing a repository's statistics\\
%\AndSc & lel\\
\WhenSc & I click on ``Loyalty"\\
\ThenSc & I can see whether other contributors stay long
\end{tabular}
\end{quote}
\end{framed}

\pagebreak
\begin{framed}
%%% TEMPLATE %%%
% Requirement (problem statement)
\subsubsection{Requirement: Link to Code's Documentation}
Link documentation from a repository's wiki with the relevant functions/lines in the project's code.\\[0.2cm]

\hrule~\\

% Feature
\noindent \Feature{\textbf{5:} Code in repository opens a hyperlink to the complete wiki documentation.}\\[0.2cm]

% Feature
\noindent \textbf{Connextra Feature}:
\begin{quote}
\begin{tabular}{rl}
\textbf{AS A}      & Code viewer\\
\textbf{SO THAT I} & understand the inner-workings of raw code.\\
\textbf{I WANT TO} & read the explanation for written code
\end{tabular}
\end{quote}


% Scenario
\noindent \textbf{Scenario}:
\begin{quote}
\begin{tabular}{rl}
\GivenSc & that I am viewing raw code in a repository\\
\WhenSc & I click on a function\\
\ThenSc & I should be linked to specific section of wiki explaining the code. 
\end{tabular}
\end{quote}


% Feature - GitHub contributor
\noindent \textbf{Connextra Feature}:
\begin{quote}
\begin{tabular}{rl}
\textbf{AS A}      & \textsf{GitHub} contributor\\
\textbf{SO THAT I} & can help new users understand the inner-workings \\
                   & of raw code.\\
\textbf{I WANT TO} & set up a hyperlink for code viewers to understand my\\
                   &  written code.
\end{tabular}
\end{quote}

% Scenario
\noindent \textbf{Scenario}:
\begin{quote}
\begin{tabular}{rl}
\GivenSc & that I am developing a wiki documentation\\
\WhenSc & I click on ``Connect Code to Wiki''\\
\AndSc & I specify which line I want to connect to\\
\ThenSc & I should be able to set up a hyperlink to the wiki documentation.
\end{tabular}
\end{quote}

\hrule~\\

\noindent \Feature{\textbf{6:} A small tool tip summary of that data type is shown (showing its methods and attributes.)}\\[0.2cm]

% Feature
\noindent \textbf{Connextra Feature}:
\begin{quote}
\begin{tabular}{rl}
\textbf{AS A}      & Code viewer\\
\textbf{SO THAT I} & understand the implementation of data types in written code.\\
\textbf{I WANT TO} & see the data type's methods \& attributes
\end{tabular}
\end{quote}

\pagebreak
% Scenario
\noindent \textbf{Scenario}:
\begin{quote}
\begin{tabular}{rl}
\GivenSc & I am observing the data type in the code\\
\WhenSc & I hover my mouse over a data structure\\
\ThenSc & a tool-tip window show display, showing the \\ 
        & attributes and methods of the data type.
\end{tabular}
\end{quote}

% Feature - GitHub contributor
\noindent \textbf{Connextra Feature}:
\begin{quote}
\begin{tabular}{rl}
\textbf{AS A}      & \textsf{GitHub} contributor\\
\textbf{SO THAT I} & I possess the tools to develop a tool-tip activated \\                    
                   & by mouse hover\\
\textbf{I WANT TO} & construct a way for users to posses a greater understanding \\
                   & of data structures in code.
\end{tabular}
\end{quote}

% Scenario
\noindent \textbf{Scenario}:
\begin{quote}
\begin{tabular}{rl}
\GivenSc & I am developing a wiki documentation\\
\WhenSc & I click on ``Make Tool-Tip''\\
\AndSc &  I specify which data structure I want to connect to\\
\ThenSc & I should be able to develop a tool-tip explaining methods\\
        &  \& attributes of the data structure.
\end{tabular}
\end{quote}
\end{framed}

% PAUL
\pagebreak
\begin{framed}
\subsubsection{Requirement: Visual View of Project}
A visual way to view a whole project.\\[0.2cm]

\hrule~\\

\noindent \Feature{\textbf{7:} UML Diagrams based on how files link with one another}\\[0.2cm]

% Feature
\noindent \textbf{Connextra Feature}:
\begin{quote}
\begin{tabular}{rl}
\textbf{AS A}      & user (\textsf{GitHub} contributor)\\
\textbf{SO THAT I} & I can gain a deeper understanding of how\\
                   &  the project calls and uses other files.\\
\textbf{I WANT TO} & find a simpler and more intuitive way to view\\
& the files in a project.
\end{tabular}
\end{quote}

% Scenario
\noindent \textbf{Scenario}:
\begin{quote}
\begin{tabular}{rl}
\GivenSc & that I am viewing a project or file\\
\WhenSc & I click on ``Model''\\
\ThenSc & I can get a diagram of which files utilise each other.\\
\AndSc & I can click on the files to open them in a new window.
\end{tabular}
\end{quote}

\hrule~\\

\noindent \Feature{\textbf{8:} A button which can collapse code into a simplified version of itself.}\\[0.2cm]

\noindent \CFeature{User (\textsf{GitHub} contributor)}{I}{can easily understand how a file works.}{get a simplified version of the current code.}

% Scenario
\noindent \textbf{Scenario}:
\begin{quote}
\begin{tabular}{rl}
\GivenSc & that I am viewing a file\\
\WhenSc & I click on ``Collapse'' next to a line of code\\
\ThenSc & the code inside the function is hidden.\\
\AndSc & is replaced with a simplified version of what is occurring\\
\WhenSc & I click on ``Collapse All''\\
\ThenSc & the code inside all functions are hidden\\
\AndSc & are replaced with a simplified version of what is occurring.
\end{tabular}
\end{quote}
\end{framed}

\pagebreak
\begin{framed}
% ###########   ROBERT   ########### 
\subsubsection{Requirement: Statistics based Contributor Search}
The ability to search for users (contributors) based on how active they are and what kind of projects they contribute to.\\[0.2cm]

\hrule~\\

\noindent \Feature{\textbf{9:} Searching for users based on how active they are.}\\[0.2cm]

% Feature
\noindent \textbf{Connextra Feature}:
\begin{quote}
\begin{tabular}{rl}
\textbf{AS A}      & User (\textsf{GitHub} author)\\
\textbf{SO THAT I} & can collaborate with more active users\\
\textbf{I WANT TO} & be able to search for users based on their average contributions
\end{tabular}
\end{quote}

% Scenario
\noindent \textbf{Scenario}:
\begin{quote}
\begin{tabular}{rl}
\GivenSc & that I am on the main page\\
\WhenSc  & I click on ``Find Users''\\
\AndSc   & I select the parameter ``Sort by User Activity'' and click ``Search''\\
\ThenSc  & I can see a filtered list of users based on their average \\ 
         & daily/weekly contributions 
\end{tabular}
\end{quote}

\hrule~\\

\noindent \Feature{\textbf{10:} Searching for users based on what kind of projects they contribute to.}\\[0.2cm]

% CFeature
\noindent \textbf{Connextra Feature}:
\begin{quote}
\begin{tabular}{rl}
\textbf{AS A}      & User (\textsf{GitHub} author)\\
\textbf{SO THAT I} & can collaborate with users who have similar interests\\
\textbf{I WANT TO} & be able to search for users based on the type of projects \\
                   & they've contributed to.
\end{tabular}
\end{quote}

% Scenario
\noindent \textbf{Scenario}:
\begin{quote}
\begin{tabular}{rl}
\GivenSc & that I am on the main page\\
\WhenSc  & I click on ``Find Users''\\
\ThenSc  & I can open up the ``Advanced Search''\\
\WhenSc  & I select one of the project types under the ``Project Categories"\\
         & subheading\\ 
\AndSc   & I click ``Search''\\
\ThenSc  & I can see a list of users who have contributed to any projects \\
         & of the relevant type
\end{tabular}
\end{quote}

\pagebreak

% ROBERT'S NEW SUB-FEATURE
\noindent \Feature{\textbf{11:} Users can sort their projects/repositories according to category.}\\[0.2cm]

% CFeature
\noindent \textbf{Connextra Feature}:
\begin{quote}
\begin{tabular}{rl}
\textbf{AS A}      & \textsf{Github} author\\
\textbf{SO THAT I} & have an alternative to sorting my projects\\
\textbf{I WANT TO} & be able to sort my projects by their category \\
\end{tabular}
\end{quote}


% Scenario
\noindent \textbf{Scenario}:
\begin{quote}
\begin{tabular}{rl}
\GivenSc & that I am viewing my own \textsf{Github} profile\\
\WhenSc  & I click on the ``Repositories'' tab\\
\AndSc   & I click on ``New''\\
\ThenSc  & there should a "Category" heading under the description text box\\
\AndSc   & I should be able to select a project category from the list
\end{tabular}
\end{quote}

% Scenario
\noindent \textbf{Scenario}:
\begin{quote}
\begin{tabular}{rl}
\GivenSc & that I am viewing my \textsf{Github} profile\\
\WhenSc  & I click on the ``Repositories" tab\\
\ThenSc  & I can open up the ``Advanced Search"\\
\WhenSc  & I select ``Sort by project category"\\
\AndSc   & I click ``Search"\\
\ThenSc  & I can see my projects sorted into columns\\ 
         & (one column for each project type)
\end{tabular}
\end{quote}
\end{framed}

\pagebreak
\documentclass[12pt]{article}

\usepackage{fullpage}
\usepackage{framed}
\usepackage{standalone}
\usepackage{amsmath}
\usepackage{tikz}

\begin{document}
\part*{Glossary}
\begin{table}[!htb]
\begin{tabular}{rl}
\noindent \textbf{Tool-Tip}: & a small message which appears when a cursor is positioned over an image,\\
                             &  text or hyperlink.\\
                             & \includegraphics[scale=0.9]{tooltip}\\
                             & \\
                             & \\
                             & \\
                             & \\
                             & \\
                             & \\
\end{tabular}
\end{table}
\end{document}

\end{document}
